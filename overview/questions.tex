\question\label{q:sidechannels}
% examgen: sidechannels:E
\begin{parts}
  \part[3] What is a side-channel attack?

  \begin{solution}
    A side channel is an unintended channel emitting information which is due 
    to physical implementation flaws and not theoretical weaknesses or forcing 
    attempts.
  \end{solution}

  \part[1] Is social engineering be a side channel?

  \begin{solution}
    No, social engineering would qualify as forcing; it is not due to the 
    hardware implementation, it is due to the human.
  \end{solution}
\end{parts}


\question[4]\label{q:sidechannels}
% examgen: sidechannels:E
Give an example of a side-channel attack and motivate why it is a side channel.

\begin{solution}
  A side channel is an unintended channel emitting information which is due 
  to physical implementation flaws and not theoretical weaknesses or forcing 
  attempts.

  (2 points) Extracting the secret key from a device by measuring energy 
  consumption or electromagnetic emissions while the device performs 
  computations using the secret key.

  (1 point) This is a side channel since it relies on a weakness in the 
  hardware implementation.
  (1 point) It is further an active attack since we might need the device to 
  perform operations on certain ciphertexts (or plaintexts).
\end{solution}


\question\label{q:sidechannels}
% examgen: sidechannels:C
Does knowledge about the hardware give any advantage? As a designer, as an 
attacker?


\question[2]\label{q:sidechannels}
% examgen: sidechannels:E
Describe an attack scenario where a side-channel is of central interest.

\begin{solution}
  The adversary is interested in learning classified information.
  They set up a device which records electromagnetic emissions to reconstruct 
  the image on a screen, thus when a target works with the classified data on 
  the computer the adversary sees the same image.
  This is a passive attack.
\end{solution}


\question[3]\label{q:sidechannels}
% examgen: sidechannels:E
Give an example of a passive side-channel attack.

\begin{solution}
  The adversary is interested in learning classified information.
  They set up a device which records electromagnetic emissions to reconstruct 
  the image on a screen, thus when a target works with the classified data on 
  the computer the adversary sees the same image.
  This is a passive attack since we only need to record.
\end{solution}


\question[3]\label{q:sidechannels}
% examgen: sidechannels:E
Given an example of an active side-channel attack.

\begin{solution}
  Extracting the secret key from a device by measuring energy consumption or 
  electromagnetic emissions while the device performs computations using the 
  secret key.
  It is an active attack since we might need the device to perform operations 
  on certain ciphertexts (or plaintexts).
\end{solution}


\question\label{q:sidechannels}
% examgen: sidechannels:E
What is a covert channel?


\question\label{q:sidechannels}
% examgen: sidechannels:E
Describe a scenario including a covert channel.


\question\label{q:sidechannels}
% examgen: sidechannels:E:C
Give an example of a covert channel and how we can prevent (or at least limit) 
it.
